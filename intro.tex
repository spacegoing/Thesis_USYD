%%
%% Template intro.tex
%%

\chapter{Introduction}
\label{cha:intro}


% lead lag 现象不是一直存在,个股趋势也非常重要

It is well known that single the price movement of an individual
stock not only depends on historical records but also highly
correlated to other
stocks~\cite{lo1990contrarian,mech1993portfolio} and may change
in a non-synchronous
manner~\cite{lo1990contrarian,brennan1993investment}. This
correlated yet asynchronous price movement is sometimes referred
to as the lead-lag relationship~\cite{hou2007industry} between a
group of stocks and is thought to arise from the different speed
of information
diffusion\cite{lo1990contrarian,badrinath1995shepherds,mcqueen1996delayed}.
When new information hits the market, some stocks react faster
than others and identification of these leading stocks and their
lead-lag relationships to other lagging stocks provides strong
predictive evidence to the latter\textquotesingle s price
movement.


Extracting informational price changes from market price data has
been a long existing challenge in stock trading industry.
Researchers have developed hundreds of technical
indicators~\cite{kirkpatrick2010technical} to recognize trend or
predict volatility in future stock price movement. We are
inspired by the significant progress in computer vision area
where researchers developed neural networks which outperform
hand-crafted features such as SIFT, HOG, and SURF. In this paper
we try to investigate a multi-task hierarchical RNN neural
networks in order to replace those hand-crafted technical
indicators.



%%% Local Variables: 
%%% mode: latex
%%% TeX-master: "thesis"
%%% End: 
