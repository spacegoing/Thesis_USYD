%%
%% Template conclusion.tex
%%

\chapter{Conclusion}
\label{cha:conclusion}


This thesis investigates two long-standing yet rarely explored
sequence learning challenges under the Probabilistic Graphical
Models (PGMs) framework: learning multi-timescale representations
on a single sequence and learning higher-order dynamics between
multi-sequences.

In Part \ref{part:1} we demonstrate the first challenge by
exploiting deep Hidden Markov Models, a type of directed PGMs,
under the Reinforcement Learning (RL) paradigm. We prove that the
Semi-Markov Decision Problem (SMDP) formulated option framework
has a Markov Decision Problem (MDP) equivalence, namely the
Skill-Action framework. To the best of our knowledge, this is the
first work proving this equivalence and giving the temporal
abstraction problem a Probabilistic Graphical Models (PGMs)
description. This is also the first work introducing the
prevalent and powerful Transformer neural networks into the
Reinforcement Learning paradigm. As shown in our
experiments~\ref{cha:sa_app}, compared to the option framework,
SA outperforms all baselines on infinite horizon environments and
is much simpler yet more effective. It has shown an exceptional
scalability and paves the way for a large scale pre-training
framework in Reinforcement Learning.

One of the most important contribution in Part \ref{part:1} is
the novel \textbf{wide value function} proposed in
\Secref{sec:sa_mdp}. Rather than use the conventional value
function $V[\rvs_t]$, we define the \textbf{skill value upon
  arrival function} as a wide value function
$V[\rvs_t,\hat{\rvo}_{t-1}]$. We also prove that the wide value
function is an unbiased estimation of the conventional value
function, and it even has a smaller variance than the
conventional one. The wide value function is the key component of
proving the equivalence between the SMDP formulated option
framework and SA. As discussed in~\Secref{sec:append_gist}, it is
also the solution to learn skills at multi-level granularities,
which is a long standing challenge in Hierarchical Reinforcement
Learning. To the best of our knowledge, this is the first work
identifying multi-level granularities of skills problem and
proposing a solution to this problem.

For the higher-order dynamics challenge, in Part \ref{part:2} we
demonstrate it by exploiting Markov Random Fields (MRFs), also
known as undirected PGMs, under the supervised learning
framework. Our main contribution in Part \ref{part:2} is
proposing an exact inference algorithm of binary MRFs with Lower
Linear Envelope Potentials (LLEPs) as higher-order energy
functions in \Secref{sec:exact_inference}. Computational
complexity of inference algorithms on MRFs with higher order
potentials grows exponentially with the scale of MRFs. By
exploiting MRFs' graphical nature, we are able to formulate the
binary MRFs into pseudo-boolean functions and solving the
inference problem under the efficient graph-cuts algorithm. We
also propose an efficient learning algorithm for MRFs with LLEPs
energy functions by employing the Latent Structural Support
Vector Machines (LSSVMs) algorithm in \Secref{sec:opt}.
Experiments in \Secref{sec:synth-check} show that the MRF-LSSVMs
framework can learn LLEPs exactly and outperforms previous works
by a large margin.

We also extend the MRF-LSSVMs framework to time series and apply
it to financial time-series data set in
\Chapref{cha:mrf_lssvm_app}. By designing a multi-task Recurrent
Neural Networks (RNNs) as a unary feature extractor, we are able
to treat each stock as a node in MRFs and represent the unary
energy as RNNs extracted embeddings. One special advantage of
financial data set is that higher-order maximum cliques are
explicitly defined by financial index companies as sector lists,
such as energy sectors and consumer sectors. By using these
industry defined sector lists as maximum cliques, we are able to
model higher-order dynamics between stocks by employing the
MRF-LSSVMs framework. We propose a sub-gradient algorithm to
perform end-to-end training of the RNN and the binary MRFs with
high-order energy functions. We conduct thorough empirical
studies on three popular Chinese stock market indexes and the
proposed method outperforms baseline approaches. To our best
knowledge, the proposed technique is the first work to
investigate higher-order dynamics with MRFs for stock price
movement prediction.

%%% Local Variables: 
%%% mode: latex
%%% TeX-master: "thesis"
%%% End: 
